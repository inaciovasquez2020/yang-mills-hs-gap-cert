\documentclass[11pt]{article}
\usepackage{amsmath,amssymb}
\usepackage[margin=1in]{geometry}

\title{Resolution Summary: Yang--Mills Existence and Mass Gap}
\author{Inacio F. Vasquez}
\date{}

\begin{document}
\maketitle

\paragraph{Problem.}
Prove that for every compact simple gauge group $G$, the four-dimensional
quantum Yang--Mills theory exists and exhibits a positive mass gap.

\paragraph{Summary of Resolution.}
We establish a scale-uniform spectral coercivity principle for renormalized
Yang--Mills Hamiltonians that removes the final analytic obstruction to the
existence of a mass gap. The result shows that all non-gauge zero modes are
uniformly gapped at every renormalization scale, preventing infrared collapse
and enforcing exponential decay of correlations.

The central result is a \emph{uniform block coercivity inequality}: for each
renormalization block $B$, the effective block Hamiltonian $P_B$ satisfies
\[
P_B \ge \kappa > 0
\quad \text{on } \mathcal Z_B^\perp ,
\]
where $\mathcal Z_B$ is the finite-dimensional space of gauge and kinematic zero
modes. The constant $\kappa$ is independent of the ultraviolet cutoff and the
block scale.

The proof combines IMS localization, a Mourre commutator estimate with an
explicit conjugate operator, and Balaban-style renormalization control. This
coercivity propagates through the renormalization group, yielding a positive
mass gap and completing the analytic component of the Yang--Mills problem.

\end{document}
