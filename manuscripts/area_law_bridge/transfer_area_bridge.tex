\documentclass[11pt]{article}

\usepackage{amsmath,amssymb,amsthm}
\usepackage{geometry}
\geometry{margin=1in}

\title{Area Law and Transfer Hamiltonian: Structural Reduction}
\author{}
\date{}

\begin{document}

\maketitle

\section{Transfer Matrix Representation}

Let $W(R,T)$ be a rectangular Wilson loop of spatial extent $R$
and Euclidean time extent $T$.

By reflection positivity and the transfer-matrix construction,

\[
\langle W(R,T) \rangle
=
\langle \Omega, \Pi_R e^{-T H} \Pi_R \Omega \rangle ,
\]

where:
\begin{itemize}
\item $H$ is the transfer Hamiltonian,
\item $\Omega$ is the vacuum,
\item $\Pi_R$ creates a static quark–antiquark pair at separation $R$.
\end{itemize}

\section{Spectral Decomposition}

By the spectral theorem, define the spectral measure

\[
\mu_R(E)
=
\| \mathbf{1}_{[E,E+dE]}(H)\Pi_R \Omega \|^2 .
\]

Then

\[
\langle W(R,T) \rangle
=
\int_0^\infty e^{-E T}\, d\mu_R(E).
\]

Define

\[
E_0(R)
=
\inf \mathrm{supp}(\mu_R).
\]

\section{Exponential Time Decay Implies Gap}

If for fixed $R$

\[
\langle W(R,T) \rangle
\le
C(R)\, e^{-mT}
\quad \text{for all } T \ge 0,
\]

then necessarily

\[
E_0(R) \ge m.
\]

Thus exponential time decay forces a spectral lower bound.

\section{Area Law}

Assume spatial area law:

\[
\langle W(R,T) \rangle
\le
e^{-\sigma R T}.
\]

Then

\[
E_0(R) \ge \sigma R.
\]

Thus static quark–pair energy grows linearly:

\[
V(R) = E_0(R) \ge \sigma R.
\]

\section{Mass Gap Requirement}

The physical mass gap requires

\[
m_{\text{gap}}
=
\inf \big( \sigma(H) \setminus \{0\} \big)
>
0.
\]

This concerns finite-energy excitations above the vacuum.

\section{Exact Missing Step}

Area law gives

\[
E_0(R) \to \infty \quad \text{as } R \to \infty.
\]

But the mass gap requires

\[
\inf_R E_0(R) > 0.
\]

Thus one must prove the identity

\[
\inf \big( \sigma(H) \setminus \{0\} \big)
=
\inf_R E_0(R).
\]

This sector-comparison identity is not established
for 4D continuum Yang–Mills.

\end{document}
