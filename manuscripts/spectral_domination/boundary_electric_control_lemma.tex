\documentclass[11pt]{article}

\usepackage{amsmath,amssymb,amsthm}
\usepackage[margin=1in]{geometry}

\title{Boundary Electric Control Lemma (Structural Target)}
\author{Inacio F. Vasquez}
\date{\today}

\newtheorem{lemma}{Lemma}
\newtheorem{definition}{Definition}
\newtheorem{theorem}{Theorem}

\begin{document}
\maketitle

\section*{Purpose}

We record a local-to-boundary control inequality intended to convert
local gauge-invariant fluctuations into boundary electric energy.
This is a structural rigidity ingredient toward uniform electric coercivity.

\section*{Setup}

Let $\Lambda_L = (\mathbb{Z}/L\mathbb{Z})^4$ be the periodic lattice.

Let $\mathcal{H}_L = L^2(\mathcal{A}_L/\mathcal{G}_L)$ be the gauge-invariant Hilbert space.

Let $E_\ell^a$ denote the electric field operator on link $\ell$,
and define the electric energy

\[
H_L^{(E)} = \sum_{\ell} \sum_{a=1}^3 (E_\ell^a)^2.
\]

Let $B_R(x) \subset \Lambda_L$ be a cubic region of side length $R$
centered at $x$.

Let $\partial B_R(x)$ denote the set of spatial links crossing the boundary
of $B_R(x)$.

\section*{Local Fluctuation Functional}

For $\psi \in \mathcal{H}_L$ define the local variance in $B_R(x)$ by

\[
\mathrm{Var}_{B_R(x)}(\psi)
=
\|\psi - \mathbb{E}_{B_R(x)} \psi\|^2,
\]

where $\mathbb{E}_{B_R(x)}$ denotes averaging over gauge orbits supported
inside $B_R(x)$.

\section*{Boundary Electric Control Target}

\begin{lemma}[Boundary Electric Control Target]
There exist constants $R \ge 1$ and $C > 0$ independent of $L$
such that for all $L$ and all gauge-invariant $\psi$,

\[
\mathrm{Var}_{B_R(x)}(\psi)
\;\le\;
C
\sum_{\ell \in \partial B_R(x)}
\sum_{a=1}^3
\|E_\ell^a \psi\|^2.
\]
\end{lemma}

\section*{Interpretation}

If true, this lemma states:

Local gauge-invariant fluctuations inside a region must be detected
by nonzero electric derivatives on the boundary of that region.

Summing over $x$ yields:

\[
\mathcal{F}_L(\psi)
\;\le\;
C'
\mathcal{E}_L(\psi),
\]

which implies uniform spectral coercivity of $H_L^{(E)}$.

\section*{Status}

This document records the inequality target only.
No proof is claimed.

\end{document}
