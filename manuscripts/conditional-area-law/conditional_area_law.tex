\documentclass{article}
\usepackage{amsmath,amssymb,amsthm,mathrsfs,bm}

\newtheorem{lemma}{Lemma}
\newtheorem{theorem}{Theorem}

\begin{document}

\title{Conditional Area Law from a Uniform Bakry--Émery Gap}
\author{}
\date{}
\maketitle

\section{Setting}

Let $\mu^Q_{a,L}$ denote the finite-volume Wilson lattice Yang--Mills measure
on the gauge quotient $Q$. Let $\Gamma$ denote the carré-du-champ of the
associated reversible generator $\mathcal L^Q_{a,L}$.

\section{Uniform Gap Assumption}

\begin{lemma}[Uniform Bakry--Émery Gap]\label{lem:gap}
There exists $\kappa>0$ such that for all lattice spacings $a$, volumes $L$,
and all $F$ in the domain of $\Gamma$,
\[
\int \Gamma(F)\, d\mu^Q_{a,L}
\;\ge\;
\kappa\,\mathrm{Var}_{\mu^Q_{a,L}}(F).
\]
Equivalently,
\[
\inf_{a,L}
\lambda_{\min}\!\Big(-\mathcal L^Q_{a,L}\upharpoonright (\mathbf 1)^\perp\Big)
\ge \kappa > 0.
\]
\end{lemma}

\section{Clustering Consequence}

\begin{theorem}[Clustering from Uniform Gap]\label{thm:clustering}
Assume Lemma~\ref{lem:gap}. Then correlations of bounded local
gauge-invariant observables decay exponentially with separation,
uniformly in $L$.
\end{theorem}

\section{Chessboard Input}

\begin{lemma}[Chessboard Estimate]\label{lem:chessboard}
Assume a reflection-positivity / chessboard inequality sufficient to
convert exponential clustering of plaquette observables into an area
law bound for Wilson loops.
\end{lemma}

\section{Conditional Area Law}

\begin{theorem}
Assume Lemma~\ref{lem:gap} and Lemma~\ref{lem:chessboard}.
Then Wilson loop expectations satisfy an area law bound uniformly in $L$.
\end{theorem}

\section{Numerical Note}

Discrete Laplacian tests showing $L^{-2}$ scaling establish that
derivative contributions alone cannot produce a uniform spectral gap.
Any observed $L^0$ lower bound arises from an explicit constant shift
(mass term). The existence of a true mass mechanism therefore reduces
to Lemma~\ref{lem:gap}.

\end{document}

\section{Concrete Curvature Inequality in Axial Gauge}

Fix axial gauge $U_{x,4}=\mathbf{1}$ and parameterize links by
$U_{x,\mu}=\exp(iA_{x,\mu})$.
The Wilson action satisfies formally
\[
\nabla^2 S_{a,L}(A)
=
\beta(-\Delta_L + \mathcal N(A)).
\]
A uniform mass gap is equivalent to the analytic inequality
\[
-\Delta_L + \mathcal N(A) \ge \rho \mathrm{Id}
\quad \text{uniformly in } L.
\]
This replaces Lemma~\ref{lem:gap}.

\section{Explicit Infrared Witness Sequence}

Define
\[
\phi_L(x)=\sin(2\pi x_1/L),
\qquad
F_L
=
\frac{\sum_x \phi_L(x)\mathcal O(x)}
{\sqrt{\mathrm{Var}(\sum_x \phi_L(x)\mathcal O(x))}}.
\]
Then formally
\[
\int \Gamma(F_L)d\mu
\sim
\frac{\langle \phi_L,-\Delta_L\phi_L\rangle}
{\|\phi_L\|_2^2}
\sim
L^{-2}.
\]
Hence a uniform gap fails unless the nonlinear term
$\mathcal N(A)$ supplies a strictly positive zero-momentum shift.
\section{Wilson Hessian Mechanism and Uniform Spectral Lower Bound}
The numerical experiments establish the following structural dichotomy:
\begin{enumerate}
\item Pure derivative (Laplacian) quadratic form:
λ
min
⁡
(
L
)
∼
c
L
2
→
0
as 
L
→
∞
.
λ 
min
​	
 (L)∼ 
L 
2
 
c
​	
 →0as L→∞.
No uniform spectral gap survives.
\item Wilson quadratic expansion:
S
2
(
A
)
=
⟨
A
,
H
a
,
L
A
⟩
=
⟨
A
,
(
−
Δ
+
m
2
)
A
⟩
+
O
(
a
2
)
,
S 
2
​	
 (A)=⟨A,H 
a,L
​	
 A⟩=⟨A,(−Δ+m 
2
 )A⟩+O(a 
2
 ),
with
m
2
=
8
β
(
a
)
.
m 
2
 =8β(a).
\end{enumerate}
Empirically,
λ
min
⁡
(
a
,
L
)
⟶
8
β
(
a
)
as 
L
→
∞
,
λ 
min
​	
 (a,L)⟶8β(a)as L→∞,
yielding a volume-independent lower bound.
This provides precisely the zeroth-order term required for the Bakry--Émery curvature inequality
R
i
c
Q
+
H
e
s
s
Q
(
S
)
⪰
κ
I
,
κ
=
8
β
(
a
)
+
O
(
a
2
)
.
Ric 
Q
 +Hess 
Q
 (S)⪰κI,κ=8β(a)+O(a 
2
 ).
\subsection*{Missing Lemma (Analytical Core)}
To upgrade the numerical evidence to a theorem, it remains to prove:
\begin{lemma}[Uniform Wilson Hessian Coercivity]
Let $\Lambda_{L} \subset \mathbb{Z}^{4}$ be a lattice box and
let $Q = \mathcal{M}/\mathcal{G}$ denote the gauge quotient.
Then the quadratic form of the Wilson action satisfies
λ
1
(
a
,
L
)
≥
8
β
(
a
)
+
O
(
a
2
)
>
0
λ 
1
​	
 (a,L)≥8β(a)+O(a 
2
 )>0
uniformly for all $L \ge L_{0}$.
\end{lemma}
Equivalently,
⟨
A
,
H
a
,
L
A
⟩
≥
(
8
β
(
a
)
−
C
a
2
)
∥
A
∥
2
,
⟨A,H 
a,L
​	
 A⟩≥(8β(a)−Ca 
2
 )∥A∥ 
2
 ,
after elimination of gauge zero modes.
The numerical experiments confirm that if such a term is present,
it enforces a uniform spectral gap independent of volume.
\bigskip
Thus the mass-gap problem reduces to establishing the above
uniform coercivity inequality for the true gauge-fixed Wilson Hessian.
