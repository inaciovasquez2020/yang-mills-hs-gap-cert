\documentclass[11pt]{article}
\usepackage{amsmath,amssymb,amsthm,hyperref}
\newtheorem{theorem}{Theorem}
\newtheorem{lemma}{Lemma}
\newtheorem{definition}{Definition}
\newtheorem{assumption}{Assumption}
\begin{document}

\title{FLUX: OS-Stable Tightness and Entropy Persistence for Lattice-to-Continuum Yang--Mills}
\author{}
\date{}
\maketitle

\begin{definition}[Admissible lattice class]
An admissible lattice theory at cutoff $\Lambda$ consists of a gauge-invariant probability space $(\Omega_\Lambda,\mathcal F_\Lambda,\mu_\Lambda)$ with OS positivity, locality, and uniformly bounded action density.
\end{definition}

\begin{assumption}[FLUX]
There exists a family of local gauge-invariant flux observables $\{\Phi_{\Lambda}(f)\}_{f\in \mathcal S}$ such that:
(i) Uniform exponential moment: $\sup_\Lambda \mathbb E_{\mu_\Lambda}[\exp(\alpha|\Phi_\Lambda(f)|)]<\infty$ for some $\alpha>0$ and all $f\in\mathcal S$.
(ii) OS-stable reflection structure: for all admissible $A,B$ supported in opposite time half-spaces,
$\langle A, \Theta B\rangle_{\mu_\Lambda}$ is well-defined and compatible with $\Phi_\Lambda$-generated cylinders.
(iii) Separating family: the joint laws of $\{\Phi_\Lambda(f_i)\}_{i\le m}$ separate subsequential limits.
\end{assumption}

\begin{theorem}[Tightness from FLUX]
Under FLUX, the family $\{\mu_\Lambda\}$ is tight in the topology induced by the cylinder $\sigma$-algebra generated by flux observables.
\end{theorem}

\begin{lemma}[OS-stable lower semicontinuity of orbit entropy]
Let $\mu_{\Lambda_n}\Rightarrow \mu$ along a subsequence in the FLUX topology. If OS positivity is stable along the subsequence, then the orbit-entropy functional $\mathcal E$ is lower semicontinuous:
\[
\mathcal E(\mu)\ge \liminf_{n\to\infty}\mathcal E(\mu_{\Lambda_n}).
\]
\end{lemma}

\begin{theorem}[Final Wall, publication form]
Any OS-positive nonabelian Yang--Mills continuum limit arising from an admissible lattice class satisfying FLUX has irreducible orbit entropy and therefore cannot satisfy Gaussian domination.
\end{theorem}

\end{document}
