\documentclass[11pt]{article}
\usepackage[margin=1in]{geometry}
\usepackage{amsmath,amssymb,amsthm}
\usepackage{hyperref}

\newtheorem{theorem}{Theorem}
\newtheorem{definition}{Definition}
\newtheorem{claim}{Claim}

\title{Nonabelian Domination Wall:\\A Structural Obstruction to Gaussian Reduction in YM$_4$}
\author{Inacio F. Vasquez}
\date{2026}

\begin{document}
\maketitle

\begin{abstract}
We formalize the Nonabelian Domination Wall: a structural obstruction to
reducing four-dimensional nonabelian Yang--Mills theory to a massive Gaussian
reference measure via domination or infrared inequalities compatible with
reflection positivity, gauge invariance, and bounded admissibility.
\end{abstract}

\section{Domination Target}

A domination inequality would assert existence of $m>0$, $C<\infty$ such that
\[
\langle F\rangle_{\mathrm{YM}} \le C \langle |F| \rangle_{\mathrm{Gauss},m}
\]
for all gauge-invariant observables $F$ supported in a slab.

\section{Structural Obstruction}

Nonabelian gauge orbit volume and Gribov degeneracy prevent uniform control
by any Gaussian reference without violating OS positivity or admissibility.

\section{Status}

This document registers the NDW as a candidate final wall. No domination
inequality of the above form is known or expected to hold in YM$_4$.
\end{document}
