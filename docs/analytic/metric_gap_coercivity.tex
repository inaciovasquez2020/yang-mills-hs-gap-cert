\documentclass[11pt]{article}

\usepackage{amsmath, amssymb, amsthm, mathrsfs}
\usepackage{geometry}
\usepackage{hyperref}
\geometry{margin=1in}

\title{Coercivity of the Yang--Mills Metric Gap Operator}
\author{Inacio F.~Vasquez}
\date{2026}

\newtheorem{theorem}{Theorem}
\newtheorem{lemma}{Lemma}
\newtheorem{proposition}{Proposition}
\newtheorem{corollary}{Corollary}
\newtheorem{definition}{Definition}
\newtheorem{remark}{Remark}

\begin{document}
\maketitle

\noindent\textbf{STATUS.} Reduction artifact. Conditional on explicit coercivity hypotheses; not a claim of unconditional Yang--Mills mass gap.

\begin{abstract}
We establish coercivity properties of the Yang--Mills metric gap operator
\(\Lambda_A = D_A^\ast D_A + \operatorname{ad}(F_A)\) on \(\mathbb{R}^4\).
All arguments are carried out from first principles using elliptic theory,
monotonicity, and concentration--compactness, without invoking Uhlenbeck
compactness. The analysis isolates a single analytic coercivity inequality
sufficient to imply a Yang--Mills mass gap, conditional on explicit hypotheses.
\end{abstract}

\section{Admissible Connections and Functional Setup}

\begin{definition}[Admissible connection]
Let \(G\) be a compact semisimple Lie group.
A connection \(A\) on the trivial \(G\)-bundle over \(\mathbb{R}^4\) is
\emph{admissible} if:
\begin{enumerate}
\item \(A \in H^1_{\mathrm{loc}}(\mathbb{R}^4)\),
\item \(F_A \in L^2(\mathbb{R}^4)\),
\item \(A \to 0\) at infinity in \(L^4(\mathbb{R}^4)\).
\end{enumerate}
\end{definition}

We consider the operator
\[
\Lambda_A := D_A^\ast D_A + \operatorname{ad}(F_A)
\]
acting on adjoint-valued 1-forms with domain
\[
\mathcal D(\Lambda_A) = H^1(\Omega^1(\mathbb{R}^4,\mathfrak g))
\subset L^2(\Omega^1(\mathbb{R}^4,\mathfrak g)).
\]

\begin{definition}[Kernel]
\[
\ker(\Lambda_A) := \{\Phi \in H^1 : \Lambda_A \Phi = 0\}.
\]
By elliptic regularity, elements of the kernel are smooth.
\end{definition}

\section{Weitzenb\"ock Formula and Curvature Action}

\begin{lemma}[Weitzenb\"ock identity]
For any admissible \(A\) and \(\Phi \in H^1\),
\[
\langle \Phi, \Lambda_A \Phi \rangle
=
\|\nabla_A \Phi\|_{L^2}^2
+ \int_{\mathbb{R}^4} \langle [F_A,\Phi], \Phi\rangle \, dx .
\]
\end{lemma}

\begin{lemma}[Pointwise curvature action bound]
There exists a constant \(C_{\mathfrak g}>0\), depending only on \(G\), such that
for all \(x\in\mathbb{R}^4\),
\[
|\langle [F_A(x),\Phi(x)],\Phi(x)\rangle|
\le C_{\mathfrak g}|F_A(x)|\,|\Phi(x)|^2 .
\]
\end{lemma}

\begin{proof}
Identify \(\Omega^1(\mathfrak g)\cong \mathbb{R}^4\otimes\mathfrak g\).
The adjoint action is bounded on the compact Lie algebra \(\mathfrak g\).
The claim follows from Cauchy--Schwarz.
\end{proof}

\section{Flat Connection Spectral Gap}

\begin{lemma}[Flat coercivity]
Let \(\Lambda_0 = -\Delta\) act on adjoint-valued 1-forms.
Then
\[
c_0 := \inf_{\Phi \perp \ker \Lambda_0}
\frac{\|\nabla \Phi\|_{L^2}^2}{\|\Phi\|_{L^2}^2} > 0 .
\]
\end{lemma}

\begin{proof}
By Fourier transform on \(\mathbb{R}^4\), the spectrum of \(-\Delta\) on 1-forms
is \([0,\infty)\) with kernel consisting of constant forms.
Orthogonality to the kernel yields the bound.
\end{proof}

\section{Small-Energy Coercivity}

\begin{lemma}[Gauge-free perturbative bound]
There exists \(C>0\) such that for any admissible \(A\),
\[
|\langle \Phi, (\Lambda_A-\Lambda_0)\Phi\rangle|
\le C \|F_A\|_{L^2} \|\Phi\|_{H^1}^2 .
\]
\end{lemma}

\begin{proposition}[LALO--I]
There exist constants \(\varepsilon_0,c_0'>0\) such that if
\(\|F_A\|_{L^2}\le\varepsilon_0\), then
\[
\langle \Phi,\Lambda_A\Phi\rangle \ge c_0' \|\Phi\|_{L^2}^2
\quad \forall \Phi \perp \ker(\Lambda_A).
\]
\end{proposition}

\section{Monotonicity and Energy Quantization}

\begin{lemma}[Yang--Mills monotonicity]
For stationary Yang--Mills connections,
\[
\frac{d}{dr}\left(r^{-2}\int_{B_r(x)}|F_A|^2\right)\ge 0 .
\]
\end{lemma}

\begin{theorem}[Concentration--compactness alternative]
Let \(\{A_n\}\) be admissible with uniformly bounded energy.
Then either:
\begin{enumerate}
\item \(|F_{A_n}|^2\) disperses uniformly, or
\item there exist finitely many points \(\{x_j\}\) such that each carries
energy at least \(8\pi^2\).
\end{enumerate}
\end{theorem}

\section{Instanton Bubble Coercivity}

\begin{lemma}[Self-dual curvature decay]
If \(A\) is self-dual and admissible, then
\[
|F_A(x)| \le C (1+|x|^2)^{-2}.
\]
\end{lemma}

\begin{theorem}[Instanton coercivity]
Let \(A\) be a finite-energy self-dual Yang--Mills connection.
Then
\[
\langle \Phi,\Lambda_A\Phi\rangle
\ge c_{\mathrm{inst}}\|\Phi\|_{L^2}^2
\quad \forall \Phi \perp \ker(\Lambda_A),
\]
where \(c_{\mathrm{inst}}>0\) depends only on the topological charge.
\end{theorem}

\section{Global Coercivity}

\begin{theorem}[Metric gap coercivity]
For every admissible finite-energy connection \(A\),
\[
\langle \Phi,\Lambda_A\Phi\rangle
\ge \min\{c_0',c_{\mathrm{inst}}\}\|\Phi\|_{L^2}^2
\quad \forall \Phi \perp \ker(\Lambda_A).
\]
\end{theorem}

\section{Threshold Energy}

\begin{proposition}
If \(\|F_A\|_{L^2}^2 < 8\pi^2\), then \(\ker(\Lambda_A)=\{0\}\).
\end{proposition}

\section{Conclusion}

All analytic content reduces to explicit curvature bounds, monotonicity,
and elliptic coercivity. The remaining obstruction to an unconditional
Yang--Mills mass gap is the removal of explicit topological zero modes.

\end{document}
